\documentclass[10pt]{beamer}
\usepackage[T1,T2A]{fontenc}
\usepackage[utf8]{inputenc}
\usepackage{hyperref}
\hypersetup{unicode=true}
\usepackage{fontawesome}
\usepackage{graphicx}
\usepackage[english,russian]{babel}

\usepackage[T1]{fontenc}
\usepackage{fontawesome}
\usepackage{PTSans} 
\mode<presentation>
{
  \usetheme[progressbar=foot,numbering=fraction,background=light]{metropolis} 
  \usecolortheme{default}
  \usefonttheme{default}
  \setbeamertemplate{navigation symbols}{}
  \setbeamertemplate{caption}[numbered]
} 

\let\textttorig\texttt
\renewcommand<>{\texttt}[1]{%
  \only#2{\textttorig{#1}}%
}

\usepackage{minted}

\usepackage{xcolor}
\definecolor{codecolor}{HTML}{FFC300}

\usepackage{tcolorbox}
\tcbuselibrary{most,listingsutf8,minted}

\tcbset{tcbox width=auto,left=1mm,top=1mm,bottom=1mm,
right=1mm,boxsep=1mm,middle=1pt}

\newtcblisting{myr}[1]{colback=codecolor!5,colframe=codecolor!80!black,listing only, 
minted options={numbers=left, style=tcblatex,fontsize=\tiny,breaklines,autogobble,linenos,numbersep=3mm},
left=5mm,enhanced,
title=#1, fonttitle=\bfseries,
listing engine=minted,minted language=r}

\definecolor{mygreen}{HTML}{37980D}
\definecolor{myblue}{HTML}{0D089F}
\definecolor{myred}{HTML}{98290D}

\usepackage{listings}

\lstdefinelanguage{XML}
{
  morestring=[b]",
  morecomment=[s]{<!--}{-->},
  morestring=[s]{>}{<},
  morekeywords={ref,xmlns,version,type,canonicalRef,metr,real,target}
}

\lstdefinestyle{myxml}{
language=XML,
showspaces=false,
showtabs=false,
basicstyle=\ttfamily,
columns=fullflexible,
breaklines=true,
showstringspaces=false,
breakatwhitespace=true,
escapeinside={(*@}{@*)},
basicstyle=\color{mygreen}\ttfamily,
stringstyle=\color{myred},
commentstyle=\color{myblue}\upshape,
keywordstyle=\color{myblue}\bfseries,
}

\title{Приложение <<Qt File Manager>>}
\subtitle{Отчет о проектной работе по курсу <<Основы информатики и программирования>>}
\author{Д. А. Куусела}
\date{8 июня 2021}

\begin{document}

\maketitle

\begin{frame}[fragile]{Цель}
    \begin{block}{Цель работы}

        Разработать простой файловый менеджер на C++ и QtQuick
    \end{block}
    \begin{block}{Задачи}
        \begin{enumerate}
            \item Реализовать работу с файлами и директориям на C++
            \item Организовать хранение данных о файлах
            \item Разработать интерфейс приложения
            \item Разработать функции для загрузки информации в QML
        \end{enumerate}
    \end{block}
\end{frame}

\begin{frame}{Тестовая директория}
    Тестовая структура директорий:\\
    \begin{center}
        \includegraphics[scale=0.5]{images/Screenshot_20210609_150249.png}
    \end{center}
    <<.>> -- текущая директория, <<..>> -- родительская директория (для UNIX систем, я тестировал на MacOS).\\
    Файлы имена, которых начинаются с точки (<<.hidden>>) -- скрытые файлы, обычно они не показываются. Текущая версия приложения показывает все файлы, включая скрытые.
\end{frame}

\begin{frame}
    \frametitle{Этапы разработки приложения}
    \begin{itemize}
        \item Реализация класса (Utils) для работы с директориям и файлами (получение информации о типе, размере и др.)
        \item Разработка QML интерфейса
        \item Реализация отображения информации о файлах в QML (черех GridView)
        \item Доработка класса Utils, добавление функций для перехода по директориям
    \end{itemize}
\end{frame}

\begin{frame}[fragile]{Определение типа файла}
    Чтобы определить является объект файлом или директорий, используются функции Qt: QFileInfo.isFile и QFileInfo.isDir. Если объект является файлом, то обрабатывается его MIME-тип:
    \begin{minted}{C++}
QString Utils::getMimeName(QString path) {
    QMimeDatabase db;
    QMimeType fileType = db.mimeTypeForFile(path,
                        QMimeDatabase::MatchExtension);
    return fileType.name();
}
bool Utils::isVideoFile(QString path) {
    return getMimeName(path).startsWith("video");
}
bool Utils::isCodeFile(QString path) {
    // с text/x начинаются файлы с исходным кодом
    return getMimeName(path).startsWith("text/x");
}
// и другие функции...
    \end{minted}
\end{frame}

\begin{frame}{Результат}
    Приложение показывает информацию о файлах и позволяет переходить в директории.
    \begin{center}
        \includegraphics[width=0.7\textwidth]{images/Screenshot_20210609_142554.png}
    \end{center}
\end{frame}

\begin{frame}{Результат}
    Если имя файла слишком длинное, то при выборе иконки текст раскрывается.
    \begin{center}
        \includegraphics[width=0.7\textwidth]{images/Screenshot_20210609_142546.png}
    \end{center}
\end{frame}

\begin{frame}[fragile]{Результат}
    Реализован запланированный функционал, получено приложение файловый менеджер. В будущем можно добавить такие фичи:
    \begin{itemize}
        \item Создание новых папок и файлов.
        \item Настройка для скрытия <<точка-файлов>>.
        \item Разные режимы сортировки файлов.
    \end{itemize}
\end{frame}

\begin{frame}
    \begin{center}
        \Large
        Спасибо за внимание!
    \end{center}
\end{frame}

\end{document}
