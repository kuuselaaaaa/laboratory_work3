%%% Для сборки выполнить 2 раза команду: pdflatex <имя файла>

\documentclass[a4paper,12pt]{article}

\usepackage{ucs}
\usepackage[utf8x]{inputenc}
\usepackage[russian]{babel}
%\usepackage{cmlgc}
\usepackage{graphicx}
\usepackage{listings}
\usepackage{xcolor}
\usepackage{titlesec}
%\usepackage{courier}

\makeatletter
\renewcommand\@biblabel[1]{#1.}
\makeatother

\newcommand{\myrule}[1]{\rule{#1}{0.4pt}}
\newcommand{\sign}[2][~]{{\small\myrule{#2}\\[-0.7em]\makebox[#2]{\it #1}}}

% Поля
\usepackage[top=20mm, left=30mm, right=10mm, bottom=20mm, nohead]{geometry}
\usepackage{indentfirst}

% Межстрочный интервал
\renewcommand{\baselinestretch}{1.50}

% ------------------------------------------------------------------------------
% minted
% ------------------------------------------------------------------------------
\usepackage{minted}


% ------------------------------------------------------------------------------
% tcolorbox / tcblisting
% ------------------------------------------------------------------------------

%%%%%%%%%%%%%%%%%%%%%%%%%%%%%%%%%%%%%%%

\begin{document}

%%%%%%%%%%%%%%%%%%%%%%%%%%%%%%%
%%%                         %%%
%%% Начало титульного листа %%%

\thispagestyle{empty}
\begin{center}


    \renewcommand{\baselinestretch}{1}
    {\large
        {\sc Петрозаводский государственный университет\\
            Институт математики и информационных технологий\\
            Кафедра информатики и математического обеспечения
        }
    }

\end{center}


\begin{center}
    %%%%%%%%%%%%%%%%%%%%%%%%%
    %
    % Раскомментируйте (уберите знак процента в начале строки)
    % для одной из строк типа направления  - бакалавриат/
    % магистратура и для одной из
    % строк Вашего направление подготовки
    %
    Направление подготовки бакалавриата \\
    % 01.03.02 --- Прикладная математика и информатика \\
    % 09.03.02 --- Информационные системы и технологии \\
    09.03.04 --- Программная инженерия \\
    %%%%%%%%%%%%%%%%%%%%%%%%%
\end{center}

\vfill

\begin{center}
    {\normalsize
    Отчет о проектной работе по курсу <<Основы информатики и программирования>>}

    \medskip

    %%% Название работы %%%
    {\Large \sc {Приложение <<Qt File Manager>> }} \\
\end{center}

\medskip

\begin{flushright}
    \parbox{11cm}{%
        \renewcommand{\baselinestretch}{1.2}
        \normalsize
        Выполнил:\\
        % Выполнили:\\
        %%% ФИО студента %%%
        студент 1 курса группы 22107
        \begin{flushright}
            Д. А. Куусела \sign[подпись]{4cm}
        \end{flushright}

        %%% Второй участник %%%
        % студента 1 курса группы 2210X
        % \begin{flushright}
        % 	И. О. Фамилия \sign[подпись]{4cm}
        % \end{flushright}

        %%%%%%%%%%%%%%%%%%%%%%%%%
        % девушкам применять "Выполнила" и "студентка"
        %%%%%%%%%%%%%%%%%%%%%%%%%
    }
\end{flushright}

\vfill

\begin{center}
    \large
    Петрозаводск --- 2021
\end{center}

%%% Конец титульного листа  %%%
%%%                         %%%
%%%%%%%%%%%%%%%%%%%%%%%%%%%%%%%

%%%%%%%%%%%%%%%%%%%%%%%%%%%%%%%%
%%%                          %%%
%%% Содержание               %%%

\newpage

\tableofcontents

%%% Содержание              %%%
%%%                         %%%
%%%%%%%%%%%%%%%%%%%%%%%%%%%%%%%

%%%%%%%%%%%%%%%%%%%%%%%%%%%%%%%%
%%%                          %%%
%%% Введение                 %%%

%%% В введении Вы должны описать предметную область, с которой связана %%%
%%% Ваша работа, показать её актуальность, вкратце определить цель     %%%
%%% разработки					       %%%


\newpage
\section*{Введение}
\addcontentsline{toc}{section}{Введение}

Цель проекта: разработать простой файловый менеджер на C++ и QtQuick.

Задачи проекта:
\begin{enumerate}
    \item Научиться считывать содержимое директорий при помощи C++
    \item Научиться получать информацию о файлах и директориях (размер и дату изменения)
    \item Организовать хранение данных о файлах
    \item Разработать интерфейс приложения
    \item Разработать функции для загрузки информации в QML
    \item Реализовать переход по директориям
\end{enumerate}

%%%                          %%%
%%%%%%%%%%%%%%%%%%%%%%%%%%%%%%%%

%%%%%%%%%%%%%%%%%%%%%%%%%%%%%%%
%%% Требования к приложению %%%
% \newpage
\section{Требования к приложению}
\begin{enumerate}
    \item Возможность просматривать содержимое директорий
    \item Возможность переходить в директорию по клику
    \item Возможность открыть файл в приложении по умолчанию
\end{enumerate}
% \subsection{Подраздел}

%%%                                     %%%
%%%%%%%%%%%%%%%%%%%%%%%%%%%%%%%%%%%%%%%%%%%

%%%%%%%%%%%%%%%%%%%%%%%%%%%%%%%%%%%%%%%%%%%
%%%                                     %%%
%%% Проектирование приложения           %%%
% \newpage
\section{Проектирование приложения}
\subsection{main.cpp}
Главный C++ файл, запускает создаёт QGuiApplication и передаёт в контекст приложения модуль Utils.
\begin{minted}{cpp}
// добавляем Utils в QML
Utils utils;
engine.rootContext()->setContextProperty("utils", &utils);
\end{minted}
\subsection{utils.cpp/utils.h}
Класс Utils. Содержит функции для получения и хранения информации о файлах. Также реализует функции для изменения текущей рабочей директории (переход в директорию). Конструктор класса Utils:
\begin{minted}{cpp}
Utils::Utils(QObject *parent) : QObject(parent)
{
    locale = QLocale::system();
    setPath(QDir::currentPath());
    readContentInfo();
}
\end{minted}
Основные функции и переменные класса:
\begin{enumerate}
    \item QHash<QString, QFileInfo> currentContent -- информация о файлах в текущей директории
    \item QList<QString> getDirContent(QString path) -- получает список файлов (абсолютные пути) в папке
    \item readContentInfo -- записывает содержимое текущей директории в currentContent
    \item void requestUpdate() -- публичный слот для запроса из QML на обновление данных
    \item bool isFile(QString path), bool isDir(QString path), bool isAudioFile(QString path) и др. -- функции для определения типа файла
    \item bool setPath(QString newPath) -- переход по пути newPath, если он существует
    \item bool openFile(QString path) -- открытие файла в приложении по умолчанию (например VLC для .mp4 файла)
\end{enumerate}
\subsection{BottomBar.qml}
Нижняя панель приложения с информацией.
\subsection{DirEntry.qml}
Элемент для отображения файла. Содержит иконку, имя и другие данные.
\subsection{main.qml}
Основной код интерфейса. Отображает список файлов (DirEntry) в виде GridView.
% \begin{enumerate}
%     \item main.cpp -- главный C++ файл, запускает создаёт QGuiApplication и передаёт в контекст приложения модуль Utils.
%     \item utils.cpp/utils.h -- класс Utils. Содержит функции для получения и хранения информации о файлах. Также реализует функции для изменения текущей рабочей директории (переход в директорию).\\Основные функции и переменные класса:
%           \begin{itemize}
%               \item QHash<QString, QFileInfo> currentContent -- информация о файлах в текущей директории
%               \item QList<QString> getDirContent(QString path) -- получает список файлов (абсолютные пути) в папке
%               \item readContentInfo -- записывает содержимое текущей директории в currentContent
%               \item void requestUpdate() -- публичный слот для запроса из QML на обновление данных
%               \item bool isFile(QString path), bool isDir(QString path), bool isAudioFile(QString path) и др. -- функции для определения типа файла
%               \item bool setPath(QString newPath) -- переход по пути newPath, если он существует
%               \item bool openFile(QString path) -- открытие файла в приложении по умолчанию (например VLC для .mp4 файла)
%           \end{itemize}
%     \item BottomBar.qml -- нижняя панель приложения с информацией.
%     \item DirEntry.qml -- элемент для отображения файла. Содержит иконку, имя и другие данные.
%     \item main.qml -- основной код интерфейса. Отображает список файлов (DirEntry) в виде GridView.
% \end{enumerate}

%%%                          %%%
%%%%%%%%%%%%%%%%%%%%%%%%%%%%%%%%

%%%%%%%%%%%%%%%%%%%%%%%%%%%%%%%%
%%%                          %%%
%%% Реализация приложения    %%%
% \newpage
\section{Реализация приложения}
Для работы с файлами и директориями используются библиотеки Qt:
\begin{itemize}
    \item QDir
    \item QDirIterator
    \item QFile
    \item QFileInfo
    \item QMimeType
    \item QMimeDatabase
\end{itemize}

Для форматирования дат и приведения размера файла в human readable формат используются библиотеки QDateTime и QLocale.

Для открытия файла в приложении по умолчанию используются QDesktopServices и QUrl.

В итоге приложение содержит:
\begin{itemize}
    \item 5 модулей (2 C++ и 3 QML)
    \item 1 C++ класс с 19 функциями
    \item 150 строк C++ кода и 308 строк QML кода
\end{itemize}

%%%                          %%%
%%%%%%%%%%%%%%%%%%%%%%%%%%%%%%%%

%%%%%%%%%%%%%%%%%%%%%%%%%%%%%%%%
%%%                          %%%
%%% Заключение               %%%
\newpage
\section*{Заключение}
\addcontentsline{toc}{section}{Заключение}
В результате реализован простой файловый менеджер, который позволяет просматривать содержимое пользовательских директорий. Приложение различает аудио, видео, фото, а также файлы с иходным кодом, для разных видов файлов загружаются разные иконки. Реализованы все запланированные функции.

Скриншот приложения:
\begin{center}
    \includegraphics[scale=0.8]{images/Screenshot_20210609_142546.png}
\end{center}

%%%                          %%%
%%%%%%%%%%%%%%%%%%%%%%%%%%%%%%%%

%%%%%%%%%%%%%%%%%%%%%%%%%%%%%%%%
%%%                          %%%
%%% Приложение               %%%

\newpage
\appendix
%\section*{Приложение}
%\addcontentsline{toc}{section}{Приложение}
%\titleformat{\section}[display]
%  {\normalfont\Large\bfseries}
%  {Приложение\ \thesection}
%  {0pt}{\Large\centering}
%\renewcommand{\thesection}{\Asbuk{section}}

%%%                          %%%
%%%%%%%%%%%%%%%%%%%%%%%%%%%%%%%%
\end{document}

